% Created 2015-09-07 Mon 00:02
\documentclass[aspectratio=169,presentation,bigger,fleqn,t]{beamer}
\usepackage[utf8]{inputenc}
\usepackage[T1]{fontenc}
\usepackage{fixltx2e}
\usepackage{graphicx}
\usepackage{grffile}
\usepackage{longtable}
\usepackage{wrapfig}
\usepackage{rotating}
\usepackage[normalem]{ulem}
\usepackage{amsmath}
\usepackage{textcomp}
\usepackage{amssymb}
\usepackage{capt-of}
\usepackage{hyperref}
\usetheme[frametitleformat=lowercase,numbering=none]{m}
\setbeamerfont{frametitle}{shape=\upshape}
\usepackage{FiraSans}
\usepackage{setspace}
\setstretch{1.3}
\usepackage{booktabs}
\hypersetup{colorlinks=true,linkcolor=blue,urlcolor=blue}
\usetheme{default}
\author{John Henderson}
\date{\today}
\title{Emacs, org-mode, and R for sciency stuff}
\hypersetup{
 pdfauthor={John Henderson},
 pdftitle={Emacs, org-mode, and R for sciency stuff},
 pdfkeywords={},
 pdfsubject={},
 pdfcreator={Emacs 24.5.1 (Org mode 8.3beta)}, 
 pdflang={English}}
\begin{document}

\maketitle



\begin{frame}[label={sec:orgheadline1}]{the life of a product developer}
Stuff I do in the course of my work (probably not that different from you!):

\begin{columns}
\begin{column}{0.5\columnwidth}
\begin{block}{direct}
\begin{itemize}
\item write up experimental plans
\item do experiments
\item collect/analyze data
\item writeup reports/presentations
\item meet with tech experts
\end{itemize}
\end{block}
\end{column}

\begin{column}{0.5\columnwidth}
\begin{block}{indirect}
\begin{itemize}
\item record all sorts info
\item meeting notes
\item todos
\item store contact info/notes
\item what to work on and when
\end{itemize}
\end{block}
\end{column}
\end{columns}
\end{frame}

\begin{frame}[fragile,label={sec:orgheadline2}]{but what the hell does emacs have to do with it?}
 Believe it or not, I \emph{learned Emacs} for \texttt{org-mode}. To date, it's the \emph{only} solution I'm
aware of that allows for all of the following in one place:
\begin{itemize}
\item notes
\item todos/time stamping/deadlines
\item tags
\item embedded code + execution
\item export to multiple formats, with images, links, table of contents, automatically
generated code blocks and/or results\ldots{}
\end{itemize}

\pause

\emph{Pretty cool!}
\end{frame}

\begin{frame}[fragile,label={sec:orgheadline3}]{some competition}
 I've always been a note taker as I like to refer to the past\ldots{} you never know what
might be useful in the future! I tried all sorts of programs:

\begin{columns}
\begin{column}{0.5\columnwidth}
\begin{block}{recording work}
\begin{itemize}
\item Word/Writer
\item \href{http://zim-wiki.org/}{zim} (personal wiki)
\item \href{https://evernote.com/}{Evernote}
\item \href{http://tiddlywiki.com/}{TiddlyWiki}
\item \href{https://www.rstudio.com/}{RStudio}?
\end{itemize}
\end{block}
\end{column}


\begin{column}{0.5\columnwidth}
\begin{block}{todo}
\begin{itemize}
\item \href{http://todotxt.com/}{\texttt{todo.txt}}
\item \href{https://en.wikipedia.org/wiki/Chandler_(software)}{Chandler}
\item \href{https://itunes.apple.com/us/app/igtd/id488595283?mt=8}{iGTD}
\item \href{http://tiddlywiki.com/}{TiddlyWiki}
\end{itemize}
\end{block}
\end{column}
\end{columns}
\end{frame}

\begin{frame}[fragile,label={sec:orgheadline4}]{ok, so what is it?}
 \texttt{Org-mode} is a major mode for the Emacs text editor.
\begin{itemize}
\item it uses markup to allow for structuring
\end{itemize}

\begin{verbatim}
* ok, so what is it?                          # heading

=Org-mode= is a major mode for the Emacs text editor.
- it uses markup to allow for structuring     # unordered list
\end{verbatim}
\end{frame}
\end{document}
